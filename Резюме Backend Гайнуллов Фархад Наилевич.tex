%-------------------------
% Резюме в LaTeX
% Автор : Гайнуллов Фархад
% Основано на: https://github.com/sb2nov/resume
% Лицензия : MIT
%------------------------

\documentclass[letterpaper,11pt]{article}

\usepackage[utf8]{inputenc}
\usepackage[T2A]{fontenc}
\usepackage[russian]{babel}
\usepackage{latexsym}
\usepackage[empty]{fullpage}
\usepackage{titlesec}
\usepackage{marvosym}
\usepackage[usenames,dvipsnames]{color}
\usepackage{verbatim}
\usepackage{enumitem}
\usepackage[hidelinks]{hyperref}
\usepackage{fancyhdr}
\usepackage{tabularx}
\input{glyphtounicode}

\pagestyle{fancy}
\fancyhf{} % очищаем все поля
\fancyfoot{}
\renewcommand{\headrulewidth}{0pt}
\renewcommand{\footrulewidth}{0pt}

% Настройка полей
\addtolength{\oddsidemargin}{-0.5in}
\addtolength{\evensidemargin}{-0.5in}
\addtolength{\textwidth}{1in}
\addtolength{\topmargin}{-.5in}
\addtolength{\textheight}{1.0in}

\urlstyle{same}

\raggedbottom
\raggedright
\setlength{\tabcolsep}{0in}

% Форматирование разделов
\titleformat{\section}{
  \vspace{-4pt}\scshape\raggedright\large
}{}{0em}{}[\color{black}\titlerule \vspace{-5pt}]

\pdfgentounicode=1

% Пользовательские команды
\newcommand{\resumeItem}[1]{
  \item\small{
    {#1 \vspace{-2pt}}
  }
}

\newcommand{\resumeSubheading}[4]{
  \vspace{-2pt}\item
    \begin{tabular*}{0.97\textwidth}[t]{l@{\extracolsep{\fill}}r}
      \textbf{#1} & #2 \\
      \textit{\small#3} & \textit{\small #4} \\
    \end{tabular*}\vspace{-7pt}
}

\newcommand{\resumeSubSubheading}[2]{
    \item
    \begin{tabular*}{0.97\textwidth}{l@{\extracolsep{\fill}}r}
      \textit{\small#1} & \textit{\small #2} \\
    \end{tabular*}\vspace{-7pt}
}

\newcommand{\resumeProjectHeading}[2]{
    \item
    \begin{tabular*}{0.97\textwidth}{l@{\extracolsep{\fill}}r}
      \small#1 & #2 \\
    \end{tabular*}\vspace{-7pt}
}

\newcommand{\resumeSubItem}[1]{\resumeItem{#1}\vspace{-4pt}}

\renewcommand\labelitemii{$\vcenter{\hbox{\tiny$\bullet$}}$}

\newcommand{\resumeSubHeadingListStart}{\begin{itemize}[leftmargin=0.15in, label={}]}
\newcommand{\resumeSubHeadingListEnd}{\end{itemize}}
\newcommand{\resumeItemListStart}{\begin{itemize}}
\newcommand{\resumeItemListEnd}{\end{itemize}\vspace{-5pt}}

\begin{document}

\begin{center}
    \textbf{\Huge \scshape Гайнуллов Фархад Наилевич} \\ \vspace{1pt}
    \small +7 (999) 199-14-01 $|$ \href{mailto:gajnullov2015@yandex.ru}{\underline{gajnullov2015@yandex.ru}} $|$ 
    \href{https://t.me/again_tdd}{\underline{Telegram: @again\_tdd}} $|$
    \href{https://github.com/IntroSpace}{\underline{github.com/IntroSpace}}
\end{center}

%-----------Краткая характеристика-----------
\section{Краткая характеристика}
  {Целеустремлённый и ответственный backend-разработчик, быстро осваиваю новые технологии, умею работать как в команде, так и самостоятельно. Стремлюсь создавать качественные, надёжные решения и постоянно развиваю свои навыки. Нацелен на результат и развитие. Умею быстро погружаться в задачи, предлагать решения и доводить их до реализации.}

%-----------ОБРАЗОВАНИЕ-----------
\section{Образование}
  \resumeSubHeadingListStart
    \resumeSubheading
      {Казанский (Приволжский) Федеральный университет -- КФУ}{Казань}
      {Институт Вычислительной Математики и Информационных Технологий, Прикладная информатика}{2028}
  \resumeSubHeadingListEnd

%-----------ОПЫТ РАБОТЫ-----------
\section{Опыт работы}
  \resumeSubHeadingListStart
    \resumeSubheading
      {Онлайн-школа IT-Compot}{Онлайн}
      {Преподаватель по программированию}{Май 2022 -- Сентябрь 2022}
      \resumeItemListStart
        \resumeItem{Учил подростков основам программирования и работе с движком Godot}
      \resumeItemListEnd

    \resumeSubheading
      {Онлайн-школа Rebotica}{Онлайн}
      {Преподаватель по алгоритмам и языку программирования Python}{Февраль 2023 -- Октябрь 2023}
      \resumeItemListStart
        \resumeItem{Преподавал основные алгоритмы, структуры данных}
        \resumeItem{Учил детей основам Python и ООП}
      \resumeItemListEnd

    \resumeSubheading
      {Ознакомительная практика}{Казань}
      {Python-разработчик}{Октябрь 2024 -- Декабрь 2024}
      \resumeItemListStart
        \resumeItem{Разработали с командой Telegram-бота, с помощью которого можно записаться в салоны красоты на необходимые услуги}
      \resumeItemListEnd
    
    \resumeSubheading
      {Хакатон КФУ в честь Нового года}{Казань}
      {Python-разработчик}{Января 2025}
      \resumeItemListStart
        \resumeItem{Разработали с командой Telegram-бота, который даёт возможность пройти разные игры с ловушками}
        \resumeItem{Команда стала победителем хакатона}
      \resumeItemListEnd
  \resumeSubHeadingListEnd

%-----------ПРОЕКТЫ-----------
\section{Проекты}
    \resumeSubHeadingListStart

      \resumeProjectHeading
          {\textbf{Мобильное приложение для изучения программирования} $|$ \emph{Java} | \href{https://github.com/IntroSpace/CoderApp}{\underline{github}}}{2020}
          \resumeItemListStart
            \resumeItem{Разработал Android-приложение для изучения программирования}
            \resumeItem{Реализовал и добавил в приложение свой язык программирования, похожий по синтаксису на Python}
          \resumeItemListEnd
        
      \resumeProjectHeading
          {\textbf{Desktop-приложение для отслеживания акций компаний} $|$ \emph{Python, PyQt5, asyncio} | \href{https://github.com/IntroSpace/diamond-stocks}{\underline{github}}}{2021}
          \resumeItemListStart
            \resumeItem{Сервис для поиска компаний, просмотра их акций}
            \resumeItem{Реализована асинхронная загрузка истории акций и информации о компании}
            \resumeItem{Возможен экспорт данных в .csv-файл или таблицу Excel}
          \resumeItemListEnd
          
      \resumeProjectHeading
          {\textbf{RPG-игра с созданием своих уровней} $|$ \emph{Python, Pygame, socket} | \href{https://github.com/IntroSpace/RPG-Diamond}{\underline{github}}}{2022}
          \resumeItemListStart
            \resumeItem{В игре реализованы разные враги, созданы стандартные уровни}
            \resumeItem{Разработан удобный редактор уровней}
            \resumeItem{Своими уровнями можно делиться по локальной сети прямо в игре}
          \resumeItemListEnd
          
      \resumeProjectHeading
          {\textbf{Социальная сеть, ориентированная на блоги} $|$ \emph{Python, Flask, sqlalchemy} | \href{https://github.com/IntroSpace/blogger_and_chatter}{\underline{github}}}{2022}
          \resumeItemListStart
            \resumeItem{Реализованы бэкенд на Flask, фронтенд на Bootstrap 5}
          \resumeItemListEnd
      \resumeProjectHeading
          {\textbf{Сайт с API для ведения учета для магазина электроники} $|$ \emph{Python, Flask, Bootstrap 5}}{2023}
          \resumeItemListStart
            \resumeItem{Реализована CRM-система для учета количества товара, бюджета, закупок/продаж}
          \resumeItemListEnd
        \resumeProjectHeading
          {\textbf{Desktop-приложение для ведения учета для магазина электроники} $|$ \emph{Python, tkinter, requests}}{2023}
          \resumeItemListStart
            \resumeItem{Приложение, в котором реализован весь функционал благодаря API сайта}
          \resumeItemListEnd
        \resumeProjectHeading
          {\textbf{Сайт-визитка для репетитора} $|$ \emph{Python, Flask, Bootstrap 5}}{2024}
          \resumeItemListStart
            \resumeItem{На сайте можно ознакомиться со всей информацией, а также 
            оставить заявку для связи}
            \resumeItem{Каждая заявка отправляется в Telegram репетитору}
          \resumeItemListEnd
        \resumeProjectHeading
          {\textbf{Telegram-бот для записи в салоны красоты} $|$ \emph{Python, pyTelegramBotAPI, sqlalchemy} | \href{https://github.com/uoyyy/salon_krasotok}{\underline{github}}}{2024}
          \resumeItemListStart
            \resumeItem{Реализован удобный функционал для записи на нужные услуги в нужное время}
            \resumeItem{Можно выбрать любой салон красоты из добавленных в бота}
          \resumeItemListEnd
        \resumeProjectHeading
          {\textbf{Telegram-бот с набором игр к Новому Году} $|$ \emph{Python, aiogram}}{2025}
          \resumeItemListStart
            \resumeItem{Реализованы игры, некоторые из которых являются тематическими к Новому Году}
          \resumeItemListEnd
        \resumeProjectHeading
          {\textbf{Планировщик задач} $|$ \emph{Python, Django} | \href{https://github.com/IntroSpace/task_manager_django}{\underline{github}}}{2025}
          \resumeItemListStart
            \resumeItem{Сделан бэкенд на Django}
            \resumeItem{У задач есть приоритет, срок выполнения, а также статус выполнено/не выполнено}
            \resumeItem{Задачи можно отфильтровать или отсортировать по разным параметрам}
          \resumeItemListEnd
        \resumeProjectHeading
          {\textbf{Система real-time аналитики правок Wikimedia} $|$ \emph{Python, FastAPI, Kafka, Redis, pydantic} | \href{https://github.com/IntroSpace/Wikimedia_Analytics}{\underline{github}}}{2025}
          \resumeItemListStart
            \resumeItem{Спроектировал и реализовал асинхронный конвейер обработки данных}
            \resumeItem{Разработал высокопроизводительный RESTful API на FastAPI для предоставления доступа к обработанной статистике, хранящейся в Redis, с валидацией моделей данных через Pydantic}
            \resumeItem{Упаковал всю инфраструктуру проекта (Kafka, Zookeeper, Redis) в контейнеры с помощью Docker и Docker Compose}
            \resumeItem{Создал консольный дашборд для визуализации данных в реальном времени}
          \resumeItemListEnd
    \resumeSubHeadingListEnd
  

%-----------Олимпиадный опыт-----------
\section{Олимпиадный опыт}
  \resumeSubHeadingListStart
    \resumeItem{Участник финала ВКОШП (Всероссийская командная олимпиада школьников по программированию)}
    \resumeItem{Участник финала командной олимпиады ПФО по информатике}
    \resumeItem{Участник финалов олимпиад МФТИ, Иннополиса и МГТУ имени Баумана по информатике}
    \resumeItem{Победитель командной олимпиады RuCode по информатике}
    \resumeItem{Призёр Регионального этапа Всероссийской олимпиады школьников по информатике}
    \resumeItem{Участник ICPC (International Collegiate Programming Contest)}
  \resumeSubHeadingListEnd

%-----------КУРСЫ-----------
\section{Курсы}
  \resumeSubHeadingListStart
    \resumeItem{Выпускник IT Школы Samsung (разработка на Java (Android и Spring-boot))}
    \resumeItem{Выпускник Яндекс.Лицея (Основы программирования на Python)}
    \resumeItem{Выпускник Яндекс.Лицея (Промышленное программирование на Python)}
  \resumeSubHeadingListEnd

%-----------ТЕХНИЧЕСКИЕ НАВЫКИ-----------
\section{Технические навыки}
 \begin{itemize}[leftmargin=0.15in, label={}]
    \small{\item{
     \textbf{Языки программирования}{: Python, C++, C\#, JavaScript, Java, SQL (PostgreSQL)} \\
     \textbf{Фреймворки}{: Django, FastAPI, PyQt5, Tkinter} \\
     \textbf{Инструменты}{: HTTP, Git, Docker, DevOps, CI/CD, RESTful API, MongoDB, Kafka, Redis, pydantic} \\
     \textbf{Алгоритмы}{: Структуры данных, Олимпиадное программирование} \\
     \textbf{Языки}{: Русский (родной), Английский (B2)}
    }}
 \end{itemize}


\end{document}